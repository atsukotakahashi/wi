\documentclass[utf8x,compress]{beamer}
\usepackage[utf8x]{inputenc}
% \DeclareUnicodeCharacter{981}{\phi}
% \usepackage{beamerthemesplit}
%\usepackage{aop-text}
%\usepackage{aop-math}

\usetheme{Malmoe}

\title{Yi}
\subtitle{An Editor in Haskell for Haskell}
\author{Jean-Philippe Bernardy}
% \email           {bernardy@chalmers.se}
\institute {Chalmers University of Technology
            % and University of Gothenburg
          }
\date {
      Haskell Symposium 2008
      \\Thursday, 25th September
      }

\begin{document}

\frame{\titlepage}

\frame{
  \frametitle{Features}
      \begin{itemize}
        \item Usable editor (more than a toy)
        \item Keymaps: vim, emacs, (cua), ...
        \item Frontends: vty, gtk, (cocoa), ...
        \item Modes: haskell, latex, perl, python, ...
        \item Static config
        \itemize{
          \item Yi is a library (as XMonad)
          \item Users "configure" Yi by combining the building blocks.
        }
      \end{itemize}
}

\frame{
  \frametitle{Architecture}
    \begin{itemize}
      \item Haskell + Hackage
      \item Purely functional core
      \item IO shell 
      \item Abstracted UIs
      \item Buffer/Editor/IO levels separated
      \item Keymaps = parsers (combinators) of input
      \item Alex-based modes
      \item Parser-based modes
    \end{itemize}
}


\frame{
  \frametitle{Haskell Support}
      \begin{itemize}
        \item Paren matching
        \item Layout rule
        \item Auto indent
        \item Ghci
        \item Cabal build
        \item (GHC API)
      \end{itemize}
}

\frame{
  \frametitle{Demo}
      \begin{itemize}
        \item Haskell Support
        \item Configuration
        \item Latex
      \end{itemize}
  
}



\end{document}